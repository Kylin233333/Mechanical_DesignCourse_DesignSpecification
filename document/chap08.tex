\chapter{键联接的选择及校核计算}
\section{高速轴键选择及校核}
\subsection{联轴器键连接计算校核}
选择的型号为A型键$b\times h \times L = 8\times 7\times 80$\\
键的工作长度$l=L-b=80-8=72mm$\\
联轴器材料为铸铁,载荷平稳取$[p] = 60 \text{ MPa}$,
则其挤压强度为:
\begin{equation}
    \sigma_p = \frac{4000T}{hld} = \frac{4000\times 35.75}{7\times 72\times 15}
    = 18.91 \text{ MPa}
\end{equation} 
$\sigma_p < [p]$,故满足强度要求

\section{中间轴键选择及校核}
\subsection{高速级大齿轮键连接计算校核}
选择的型号为双C型键$b\times h \times L = C12\times 8\times 32$,双键相隔$180^\circ$布置。\\
键的工作长度$l=1.5\times (L-0.5b)=1.5\times(32-6)=39mm$\\
齿轮材料为40Cr,载荷平稳取$[\sigma_p] = 150 \text{ MPa}$,
则其挤压强度为:
\begin{equation}
    \sigma_p = \frac{4000T}{hld} = \frac{4000\times 163.10}{8\times 39\times 17.5}
    = 119.48 \text{ MPa}
\end{equation} 
$\sigma_p < [p]$,故满足强度要求
\subsection{低速级小齿轮键连接计算校核}
选择的型号为A型键$b\times h \times L = 8\times 7\times 63$\\
键的工作长度$l=L-b=63-8=55mm$\\
齿轮材料为40Cr,载荷平稳取$[\sigma_p] = 150 \text{ MPa}$,
则其挤压强度为:
\begin{equation}
    \sigma_p = \frac{4000T}{hld} =\frac{4000\times 163.10}{8\times 55\times 17.5}= 84.72 \text{ MPa}
\end{equation} 
$\sigma_p < [p]$,故满足强度要求

\section{低速轴键选择及校核}
\subsection{低速级大齿轮键连接计算校核}
选择的型号为双C型键$b\times h \times L = C10\times 8\times 36$,双键相隔$180^\circ$布置。\\
键的工作长度$l=1.5(L-0.5b)=1.5(36-5)=46.5mm$\\
齿轮材料为40Cr,载荷平稳取$[\sigma_p] = 150 \text{ MPa}$,
则其挤压强度为:
\begin{equation}
    \sigma_p = \frac{4000T}{hld} = 125.79 \text{ MPa}
\end{equation} 
$\sigma_p < [p]$,故满足强度要求
\subsection{联轴器键连接计算校核}
选择的型号为A型键$b\times h \times L = 14\times 9\times 110$\\
键的工作长度$l=L-b=80-8=96mm$\\
联轴器材料为铸铁,载荷平稳取$[p] = 120 \text{ MPa}$,
则其挤压强度为:
\begin{equation}
    \sigma_p = \frac{4000T}{hld} = \frac{4000\times 531.35}{9\times 96\times 22.5}
    = 109.33 \text{ MPa}
\end{equation} 
$\sigma_p < [p]$,故满足强度要求

