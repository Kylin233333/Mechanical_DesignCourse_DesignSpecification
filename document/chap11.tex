\chapter{减速箱箱体附件及机体结构}
\section{减速箱附件}
\subsection{窥视孔盖和窥视孔}
减速箱顶部要开窥视孔,以便检查传动件啮合情况、润滑状况等,窥视孔应设在能看到传动零件啮合区的位置,
并有足够的大小,以便手能深入进行操作,减速器内润滑油也由窥视孔注入,为了减少油的杂质,可在窥视孔口装一过滤网。
窥视孔要有盖板,机体上开窥视孔应凸起一块,以便机械加工出支承盖板的表面并用垫片加强密封。
如图\ref{fig:11-1}所示。
\begin{figure}[!h]
    \centering
    \includegraphics[width=0.5\textwidth]{figures/顶盖.pdf}
    \caption{窥视孔及盖板示意图}
    \label{fig:11-1}
\end{figure}

\subsection{放油螺栓}
放油孔位置应在油池最低处,并安排在减速器不与其他部件靠近的一侧,以便于放油,
放油孔用螺塞堵住,因此油孔处机体外壁应凸起一块,
经机械加工成为螺塞头部的支承面,并加封油圈以加强密封。
如图\ref{fig:11-2}所示。
\begin{figure}[!h]
    \centering
    \includegraphics[angle=90,width=0.5\textwidth]{figures/放油.pdf}
    \caption{放油螺栓示意图}
    \label{fig:11-2}
\end{figure}

\subsection{油尺}
油尺常放置在便于观测减速器油面及油面稳定之处,一般多用带有螺纹部分的油尺,使用油尺时,
应使机座油尺座孔的倾斜位置便于加工和使用,油尺安置部位不能太低,以防油进入油尺座孔而溢出,
为了避免因油搅动而影响检查效果,可在油尺外装隔离套。
如图\ref{fig:11-3}所示。
\begin{figure}[!h]
    \centering
    \includegraphics[angle=90,width=0.5\textwidth]{figures/油标.pdf}
    \caption{油尺示意图}
    \label{fig:11-3}
\end{figure}

\subsection{通气器}
减速器运转时机体内温度升高,气压增大,对减速器密封极为不利,
所以在机盖顶部或窥视孔加通气器,使机体内外压力均衡。
如图\ref{fig:11-4}所示。
\begin{figure}[!h]
    \centering
    \includegraphics[angle=90,width=0.5\textwidth]{figures/放气.pdf}
    \caption{通气器示意图}
    \label{fig:11-4}
\end{figure}

\subsection{起盖螺钉}
起盖螺钉上螺纹长度应该大于机盖连接凸缘的厚度,
钉杆端部要做成圆柱、大倒角或半圆形,以免顶坏螺纹。
如图\ref{fig:11-5}所示。
\begin{figure}[!h]
    \centering
    \includegraphics[angle=90,width=0.5\textwidth]{figures/起盖.pdf}
    \caption{起盖螺钉示意图}
    \label{fig:11-5}
\end{figure}

\subsection{定位销}
为了保证剖分式机体的轴承座孔的加工和装配精度,在机体连接凸缘的长度方向两端各安置一个圆锥定位销,两销相距尽量远些,以提高定位精度
定位销直径一般取$d=(0.7 \text{ ~ } 0.8)\cdot d_2$,$d_2$为机体连接螺栓直径,长度应大于机盖和机座连接凸缘总厚度,便于装拆。
如图\ref{fig:11-6}所示。
\begin{figure}[!h]
    \centering
    \includegraphics[angle=90,width=0.5\textwidth]{figures/定位销.pdf}
    \caption{定位销示意图}
    \label{fig:11-6}
\end{figure}


\subsection{起吊装置}
为了拆卸和搬运应在机盖上装有环首螺钉或铸出吊钩,采用环首螺钉是加工工序增加,
所以本设计采用机座两端铸出吊环,机盖两端同时铸出吊耳。
如图\ref{fig:11-7}所示。
\begin{figure}[!h]
    \centering
    \includegraphics[width=0.3\textwidth]{figures/吊耳.png}
    \includegraphics[width=0.3\textwidth]{figures/吊环.png}
    \caption{起吊装置示意图}
    \label{fig:11-7}
\end{figure}
吊孔尺寸计算:
\begin{equation}
    b\approx 2.5\times \delta_{1}=2.5\times8=20mm
\end{equation}
\begin{equation}
    d=b=20mm
\end{equation}
\begin{equation}
    R=1\times d=1\times 20=20mm
\end{equation}
\begin{equation}
    B=C_1+C_2=16+14=30mm
\end{equation}
\begin{equation}
    H=0.8\cdot B=0.8\times 30=24mm
\end{equation}
\begin{equation}
    h=0.5\cdot H=0.5\times 24=12mm
\end{equation}
\begin{equation}
    r=0.25\cdot B=0.25\times 30=7.5mm
\end{equation}
吊耳尺寸计算:
\begin{equation}
    d\approx 2.5\times \delta_{1}=2.5\times8=20mm  
\end{equation}
\begin{equation}
    R=1\times d=1\times 20=20mm
\end{equation}
\begin{equation}
    e=1\times d=1\times 20=20mm
\end{equation}


\section{减速器机体结构尺寸}
减速器机体结构尺寸的确定,可按表\ref{机体结构尺寸表}进行选取。
\begin{table}[!ht]
    \centering
    \caption{减速器机体结构尺寸表}
    \begin{tabular}{|c|c|c<{\hspace{0pt}}|c<{\hspace{0pt}}|}
        \hline
        名称 & 符号 & 关系 & 数值/mm \\ 
        \hline
        机座壁厚 & $\delta$ & $0.25a+3\geq 8$ & 8 \\ 
        机盖壁厚 & $\delta_1$ & $0.02a+3\geq 8$ & 8 \\ 
        机座凸缘厚度 & $b$ & $1.5\delta$ & 12 \\ 
        机盖凸缘厚度 & $b_1$ & $1.5\delta_1$ & 12 \\ 
        机座底凸缘厚度 & $b_2$ & $2.5\delta$ & 20 \\
        地脚螺钉直径 & $d_f$ & $0.36a+12$ & M18 \\ 
        地脚螺钉数目 & $n$ & $a\leq 250$ & 4 \\ 
        轴承旁联接螺栓直径 & $d_1$ & $0.75d_f$ & M14 \\ 
        机盖与机座联接螺栓直径 & $d_2$ & $0.5 \sim 0.6$ & M10 \\
        联接螺栓$d_2$的间距 & $l$ & $15200$ & 150 \\ 
        轴承端盖螺钉直径 & $d_3$ & $0.4 \sim 0.5$ & M8 \\ 
        窥视孔盖螺钉直径 & $d_4$ & $0.3 \sim 0.4$ & M6 \\ 
        定位销直径 & $d$ & $0.7 \sim 0.8$ & 8 \\ 
        $d_f$、$d_1$、$d_2$至外机壁距离 & $c_1$ & 查表 & 24、20、16 \\ 
        $d_1$、$d_2$至凸缘边缘距离 & $c_2$ & 查表 & 18、14 \\ 
        轴承旁凸台半径 & $R_1$ & $C_2$ & 18 \\ 
        凸台高度 & $h$ & \makecell[lt]{根据低速级轴承座外径确定,\\以便于扳手操作为准} & 30 \\ 
        外箱壁至轴承座端面距离 & $l_1$ & $C_1+C_2+(5\sim 10)$ & 43 \\ 
        大齿轮顶圆与内箱壁距离 & $\Delta_1$ & $>1.2\delta$ & 10 \\ 
        齿轮端面与内箱壁距离 & $\Delta_3$ & $>\delta$ & 10 \\ 
        \makecell[t]{箱盖肋厚 \\ 箱座肋厚} & \makecell[t]{$m_1$ \\ $m$} & \makecell[t]{$m_1\approx 0.85\times\delta_1$ \\ $m \approx 0.85 \times \delta$} & \makecell[t]{6.8 \\ 6.8} \\ 
        高速轴承端盖外径 & $D_1$ & $D+(5\sim 5.5)d_3$ & 102 \\ 
        中间轴承端盖外径 & $D_2$ & $D+(5\sim 5.5)d_3$ & 102 \\ 
        低速轴承端盖外径 & $D_3$ & $D+(5\sim 5.5)d_3$ & 130 \\ 
        轴承旁联接螺栓距离 & $s$ & 尽量靠近 & 102 \\ 
        \hline
    \end{tabular}
    \label{机体结构尺寸表}
\end{table}