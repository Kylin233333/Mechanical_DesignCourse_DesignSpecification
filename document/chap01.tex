\chapter{设计任务书}
\section{设计题目}
\subsection{传动系统简图}

\begin{figure}[h]
    \centering
    \includegraphics[width=0.8\textwidth]{figures/1.png}
    \caption{传动系统简图}
    \label{fig:传动系统简图}
\end{figure}

\subsection{设计要求}

单向运转,滚筒转速允许误差±5\%,工作中短期过载不超过正常载荷的 1.4倍。(工作机效率0.96)
\begin{itemize}
    \item 传送带曳引力  $F = 5.6kN$
    \item 传送带速度 $v = 0.55m/s$
    \item 滚筒直径 $D = 360mm$
    \item 每天工作小时数 $h = 16$
    \item 工作年限 $y = 10$年(每年工作270天)
\end{itemize}
\section{设计计算步骤}
1.确定传动装置的传动方案\\
2.选择合适的电动机\\
3.计算减速器的总传动比以及分配传动比\\
4.计算减速器的动力学参数\\
5.齿轮传动的设计\\
6.开式齿轮传动设计\\
7.传动轴的设计与校核\\
8.滚动轴承的设计与校核\\
9.键联接设计\\
10.联轴器设计\\
11.减速器润滑密封设计\\
12.减速器箱体结构设计
\section{方案特点}
1.组成:传动装置由电机、联轴器、减速器、开式链条、工作机组成。\\
2.特点:齿轮相对于轴承非对称布置。\\
3.优点:结构紧凑、传动效率高、使用寿命长、维修方便。