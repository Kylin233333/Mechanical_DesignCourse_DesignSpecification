\chapter{减速箱的密封和润滑}
\section{减速箱密封}
为防止箱体内润滑剂外泄和外部杂质进入箱体内部影响箱体工作,
在构成箱体的各零件间,如箱盖与箱座间、及外伸轴的输出、输入轴与轴承盖间,
需设置不同形式的密封装置。对于无相对运动的结合面,常用密封胶、耐油橡胶垫圈等;
对于旋转零件如外伸轴的密封,则需根据其不同的运动速度和密封要求考虑不同的密封件和结构。
本设计中由于密封界面的相对速度较小,故采用接触式密封。
输入轴与轴承盖间v<3m/s,输出轴与轴承盖间也为v<3m/s,故均采用毡圈油封封油圈。
\section{齿轮的润滑}
对于大多数减速器,由于其传动件圆周速度$\mathrm{v\le12m/s}$,
通常采用浸油润滑,机体内需要有足够的润滑油,用以润滑和散热。
因为低速级小齿轮圆周速度是减速箱传动件中最大的,所以只需计算其圆周速度检验是否满足条件即可
\begin{equation}
    v=\frac{\pi d_{4} n}{60 \times 1000} 
    = \frac{\pi \times 60.81 \times 87.54}{60 \times 1000} = 0.28 m/s
\end{equation}
满足$\mathrm{v\le12m/s}$,故可采用浸油润滑。

同时为了避免油搅动时沉渣泛起,齿顶到油池底面的距离H不小于30~50mm,
这里取低速级大齿轮齿顶距箱底高度$H=60\text{mm}$,高速级大齿轮浸油深度$h$通常超过一个齿高,不小于10mm。

高速级大齿轮全齿高为 $h_c = 3.375\text{mm}$,取浸油深度$h=10\text{mm}$。

对于采用浸油润滑的多级传动,当低速级大齿轮浸油深度h1超过1/3分度圆半径时,往往会使搅油损失过大
\begin{equation}
    h_1 \le \frac{d_4}{3} = \frac{216.19}{3} \approx 72.06\text{mm}
\end{equation}
此处取$h_1 = 40\text{mm}$。

所以油的深度$H^{'}$:
\begin{equation}
    H' = H + h_1 = 60 + 40 = 100\text{mm}
\end{equation}
根据齿轮圆周速度选择工业齿轮油,牌号是L-CKD,黏度范围220-320cSt。

\section{轴承的润滑与密封}
轴承的润滑根据轴颈的速度确定,当浸油齿轮圆周速度小于$2m/s$,宜采用脂润滑;
当浸油齿轮圆周速度大于$2m/s$,可以靠机体内油的飞溅直接润滑轴承,
或引导飞溅在机体内壁上的油经机体剖分面上的油沟流到轴承进行润滑。

由上述齿轮润滑可知浸油齿轮圆周速度满足小于$2m/s$,
所以轴承采用脂润滑,并在轴承旁加挡油板,防止润滑脂流失
选用通用锂基润滑脂。

根据圆周速度小于$3m/s$,密封形式采用粗羊毛毡封油圈。
